% Clase del documento
\documentclass[12pt,twoside,titlepage]{report}





%%%%%%%%%%%%%%%%%%%%%%% Paquetes %%%%%%%%%%%%%%%%%%%%%%%

\usepackage[a4paper,bindingoffset=3mm,bottom=35mm]{geometry}


% Usad \usepackage[dvips]{graphicx} o \usepackage[pdftex]{graphicx} (no ambos)
%\usepackage[dvips]{graphicx} %%% para LaTeX. Las figuras deben estar en formato eps

\usepackage[colorlinks=true,pdftex]{hyperref}   %%% Opcional. Para incluir marcadores y enlaces en el pdf
\usepackage[pdftex]{graphicx}  %%% para pdflatex. Las figuras pueden estar en pdf, jpg, svg y otros formatos


\usepackage[spanish]{babel}

%\usepackage[latin1]{inputenc} % Usad en WinEdt/MikTex
\usepackage[utf8]{inputenc} % Usad en overleaf

%\usepackage[T1]{fontenc}


% Algunos paquetes útiles

\usepackage{amsmath,amssymb}
\usepackage{hyperref}
\usepackage{color}
\usepackage{afterpage}
\usepackage{paralist}
\usepackage{array}
\usepackage{enumerate}
\usepackage{paralist}
\usepackage{enumitem}
\usepackage{float}
\usepackage{setspace}
\usepackage{listings}
\usepackage{algorithm}
\usepackage{algorithmic}
\usepackage{fancyhdr}
\usepackage{rotating}
\usepackage{multirow}


% Otros paquetes

\usepackage{quotchap}
\usepackage{lipsum}

%%%%%%%%%%%%%%%%%%%%%%%%%%%%%%%%%%%%%%%%%%%%%%%%%%%%%%%%






%%%%%%%%%%%%%%%%%%%%%%% Definiciones básicas %%%%%%%%%%%%%%%%%%%%%%%

\newcommand{\nombreautor}{Oscar Nydza Nicpoñ}
\newcommand{\nombretutor}{Juan Manuel Serrano Hidalgo}
\newcommand{\titulotrabajo}{Comparativa entre las API de Spark en Scala y Python}
\newcommand{\escuela}{Escuela Técnica Superior\\de Ingeniería Informática}
\newcommand{\escuelalargo}{Escuela Técnica Superior de Ingeniería Informática}
\newcommand{\universidad}{Universidad Rey Juan Carlos}
\newcommand{\fecha}{Fecha}
\newcommand{\grado}{Grado en Ingeniería de Computadores}
\newcommand{\curso}{Curso 2021-2022}
\newcommand{\logoUniversidad}{logoURJC.pdf} % logoURJC.eps

%%%%%%%%%%%%%%%%%%%%%%%%%%%%%%%%%%%%%%%%%%%%%%%%%%%%%%%%%%%%%%%%%%%%






%%%%%%%%%%%%%%%%%%%%%%%%% Otras definiciones %%%%%%%%%%%%%%%%%%%%%%%%%%

% Definiciones de colores (para hidelinks)
\definecolor{BlueLink}{rgb}{0.165,0.322,0.745}
\definecolor{PinkLink}{rgb}{0.8,0.22,0.5}
\definecolor{gray}{rgb}{0.6,0.6,0.6}


% Enlaces
\hypersetup{hidelinks,pageanchor=true,colorlinks,citecolor=PinkLink,urlcolor=black,linkcolor=BlueLink}


\newcommand\blankpage{%
    \newpage
    \null
    \thispagestyle{empty}%
    %\addtocounter{page}{-1}%
    \newpage}


% Texto referencias
\addto{\captionsspanish}{\renewcommand{\bibname}{Bibliografía}}

% Texto Índice de tablas
\addto\captionsspanish{
\def\tablename{Tabla}
\def\listtablename{\'{I}ndice de tablas}
}


\floatname{algorithm}{Algoritmo}

\newfloat{algorithm}{t}{lop}


%\newenvironment{pseudocodigo}[1][htb]
%  {\renewcommand{\algorithmcfname}{Pseudocódig}% Update algorithm name
%   \begin{algorithm}[#1]%
%  }{\end{algorithm}}
  
%%%%%%%%%%%%%%%%%%%%%%%%%%%%%%%%%%%%%%%%%%%%%%%%%%%%%%%%%%%%%%%%%%%%





%%%%%%%%%%%%%%%%%%%%%%% Estilo de código (en Python) %%%%%%%%%%%%%%%%%%%%%%%

\definecolor{bg}{rgb}{0.95,0.95,0.95}
\definecolor{mydeepteal}{rgb}{0.16,0.22,0.23}
\definecolor{myteal}{rgb}{0.31,0.44,0.46}
\definecolor{mymediumteal}{rgb}{0.41,0.58,0.60}

\DeclareFixedFont{\ttb}{T1}{txtt}{bx}{n}{12} % for bold
\DeclareFixedFont{\ttm}{T1}{txtt}{m}{n}{12}  % for normal


%\newcommand*{\FormatDigit}[1]{\textcolor{mydeepteal}{#1}}
\newcommand*{\FormatDigit}[1]{\textcolor{black}{#1}}

% Python style for highlighting
\newcommand\mypythonstyle{\lstset{
language=Python,
basicstyle=\ttfamily\small,
%basicstyle=\linespread{1.0}\footnotesize\ttm,
otherkeywords={self},             % Add keywords here
keywordstyle=\bfseries\ttfamily\color{myteal},
%keywordstyle=\ttb\color{myteal},
commentstyle=\itshape\color{myteal},
stringstyle=\color{mydeepteal},
emph={MyClass,__init__},          % Custom highlighting
emphstyle=\ttb\color{mydeepteal},    % Custom highlighting style
% Any extra options here
showstringspaces=false,            %
backgroundcolor=\color{bg},
rulecolor = \color{bg},
%identifierstyle=\color{deepgreen},
breaklines=true,
numbers=left,
numbersep=5pt,
numberstyle=\tiny,
tabsize=4,
xleftmargin=1em,
frame = single,
framesep = 3pt,
framextopmargin=0pt,
framexbottommargin=0pt,
framexleftmargin=0pt,
framexrightmargin=0pt,
fontadjust=true,
basewidth=0.55em, % compactness of code
upquote=true,
}}

% Python environment
\lstnewenvironment{mypython}[1][]
{
\mypythonstyle
\lstset{#1}
}
{}

\newcommand\mypythonstylenormalinline{\lstset{
language=Python,
basicstyle=\ttfamily\normalsize,
%basicstyle=\linespread{1.0}\footnotesize\ttm,
otherkeywords={self},            % Add keywords here
keywordstyle=\bfseries\ttfamily\color{myteal},
%keywordstyle=\ttb\color{myteal},
commentstyle=\itshape\color{mymediumteal},
stringstyle=\color{mydeepteal},
emph={MyClass,__init__},          % Custom highlighting
emphstyle=\ttb\color{mydeepteal},    % Custom highlighting style
% Any extra options here
showstringspaces=false,            %
backgroundcolor=\color{bg},
rulecolor = \color{bg},
%identifierstyle=\color{deepgreen},
breaklines=false,
numbers=left,
numbersep=5pt,
numberstyle=\tiny,
tabsize=4,
xleftmargin=0em,
frame = single,
framesep = 3pt,
framextopmargin=0pt,
framexbottommargin=0pt,
framexleftmargin=0pt,
framexrightmargin=0pt,
fontadjust=true,
%basewidth=0.55em, % compactness of code
upquote=true,
}}

\newcommand\mypythoninline[1]{{\mypythonstylenormalinline\lstinline!#1!}}

%%%%%%%%%%%%%%%%%%%%%%%%%%%%%%%%%%%%%%%%%%%%%%%%%%%%%%%%%%%%%%%%%%%%%%%%%%%%%%

\usepackage{listings}

% "define" Scala
\lstdefinelanguage{scala}{
  morekeywords={abstract,case,catch,class,def,%
    do,else,extends,false,final,finally,%
    for,if,implicit,import,match,mixin,%
    new,null,object,override,package,%
    private,protected,requires,return,sealed,%
    super,this,throw,trait,true,try,%
    type,val,var,while,with,yield},
  otherkeywords={=>,<-,<\%,<:,>:,\#,@},
  sensitive=true,
  morecomment=[l]{//},
  morecomment=[n]{/*}{*/},
  morestring=[b]",
  morestring=[b]',
  morestring=[b]"""
}

\usepackage{color}
\definecolor{dkgreen}{rgb}{0,0.6,0}
\definecolor{gray}{rgb}{0.5,0.5,0.5}
\definecolor{mauve}{rgb}{0.58,0,0.82}

\lstset{frame=tb,
  language=scala,
  aboveskip=3mm,
  belowskip=3mm,
  showstringspaces=false,
  columns=flexible,
  basicstyle={\small\ttfamily},
  numbers=none,
  numberstyle=\tiny\color{gray},
  keywordstyle=\color{blue},
  commentstyle=\color{dkgreen},
  stringstyle=\color{mauve},
  frame=single,
  breaklines=true,
  breakatwhitespace=true
  tabsize=3
}


%%%%%%%%%%%%%%%%%%%%%%%%%%%% Comandos definidos por el autor 

\newcommand{\transpuesta}{\mbox{\tiny $\mathsf{T}$}}








%%%%%%%%%%%%%%%%%%%%%%%%%%%%%%%%%%%%%%%%%%%%%%%%%%%%%%%%%%%%%%%%%%%%%%%
%                           Inicio del documento                       
%%%%%%%%%%%%%%%%%%%%%%%%%%%%%%%%%%%%%%%%%%%%%%%%%%%%%%%%%%%%%%%%%%%%%%%


\begin{document}

\pagestyle{plain}




%%%%%%%%%%%%%%%%%%%%%%%%%%%%%%%%%%%% Portada %%%%%%%%%%%%%%%%%%%%%%%%%%%%%%%%%%

%\pagenumbering{gobble}
%\pagenumbering{arabic}

% Universidad, Facultad
\begin{titlepage}
\selectlanguage{spanish}


% logo
\begin{center}
    \includegraphics[scale=0.7]{\logoUniversidad}
\end{center}

\bigskip

\begin{center}
\begin{LARGE}
\escuela \\
\end{LARGE}
\end{center}

\bigskip
\bigskip

% Grado
\begin{center}
\begin{large}
\textbf{\grado}\\
\end{large}
\end{center}

% Curso
\begin{center}
\begin{large}
\textbf{\curso}\\
\end{large}
\end{center}

\bigskip

\textbf{\begin{center}
\begin{large}
\textbf{Trabajo Fin de Grado}
\end{large}
\end{center}}

\bigskip
\bigskip
\bigskip

% Nombre del TFG
\begin{center}
\textbf{\begin{large}
\MakeUppercase{\titulotrabajo}\\
\end{large}}
\end{center}

% Nombre del autor
\vspace{\fill}
\begin{center}
\textbf{Autor: \nombreautor}\\ \smallskip
% Tutor
\textbf{Tutor: \nombretutor}\\
% Añadir segundo tutor si hubiera


\bigskip

% Fecha
%\textbf{\fecha}\\
\end{center}
\end{titlepage}


%%%%%%%%%%%%%%%%%%%%%%%% Opcional %%%%%%%%%%%%%%%%%%%%%%
%\blankpage

%\thispagestyle{empty}
%\begin{center}

% Nombre del trabajo
%\textbf{\begin{large}
%\MakeUppercase{\titulotrabajo}\\*
%\end{large}}
%\vspace*{0.2cm}
%\vspace{5cm}

% Nombre del autor y del tutor
%\large Autor: \nombreautor \\* \medskip
%\large Tutor: \nombretutor \\*

%\vfill

% Escuela, universidad y fecha
%\escuelalargo \\ \smallskip
%\universidad \\
%\vspace{1cm}
%\fecha \\

%\clearpage

%\end{center}
%%%%%%%%%%%%%%%%%%%%%%%%%%%%%%%%%%%%%%%%%%%%%%%%%%%%%%%%

\hypersetup{pageanchor=true}

\normalsize
\afterpage{\blankpage} % Se deben añadir página en blanco para que lo capítulos de la memoria o estas secciones introductorias empiecen en páginas impares

%%%%%%%%%%%%%%%%%%%%%%%%%%%%%%%%%%%%%%%%%%%%%%%%%%%%%%%%%%%%%%%%%%%%%%%%%%%%%%%





% Estilo de párrafo de los capítulos
\setlength{\parskip}{0.75em}
\renewcommand{\baselinestretch}{1.25}
% Interlineado simple
\spacing{1}

\pagenumbering{Roman}
\setcounter{page}{2}


%%%%%%%%%%%%%%%%%%%%%%%%% Agradecimientos o dedicatoria %%%%%%%%%%%%%%%%%%%%%%%%%%%

\chapter*{Agradecimientos}

Breves agradecimientos o dedicatoria.

\afterpage{\blankpage}

%%%%%%%%%%%%%%%%%%%%%%%%%%%%%%%%%%%%%%%%%%%%%%%%%%%%%%%%%%%%%%%%%%%%%%%%%%%%%%%%%%%






%%%%%%%%%%%%%%%%%%%%%%%%%%%%%%%%%%%% Resumen %%%%%%%%%%%%%%%%%%%%%%%%%%%%%%%%%%%%%%

\chapter*{Resumen}

Breve resumen del Trabajo de Fin de Grado (TFG). Recomendable entre 250-300 palabras, conteniendo los principales objetivos y resultados derivados del mismo.

\mbox{} \bigskip

\noindent \textbf{Palabras clave}:
\begin{compactitem}
    \item Python
    \item Ciberseguridad
    \item Aprendizaje automático (pueden ser varias)
    \item $\ldots$
\end{compactitem}

\afterpage{\blankpage}

%%%%%%%%%%%%%%%%%%%%%%%%%%%%%%%%%%%%%%%%%%%%%%%%%%%%%%%%%%%%%%%%%%%%%%%%%%%%%%%%%%%





%%%%%%%%%%%%%%%%%%%%%%%%%%%%%%%%%%%% Índices %%%%%%%%%%%%%%%%%%%%%%%%%%%%%%%%%%%%

% Estilo de párrafo de los Índices
\setlength{\parskip}{1pt}
\renewcommand{\baselinestretch}{1}
\renewcommand{\contentsname}{Índice de contenidos}


% Índice de contenidos
\tableofcontents
\afterpage{\blankpage}

% Índice de tablas (OPCIONAL)
%\listoftables
%\afterpage{\blankpage}
%\addcontentsline{toc}{chapter}{\noindent \listtablename}

% Índice de figuras (OPCIONAL)
\listoffigures
\afterpage{\blankpage}
\addcontentsline{toc}{chapter}{\listfigurename}

% Índice de códigos/algoritmos (OPCIONAL).   El término "Códigos" se puede cambiar por "Métodos", "Funciones", "Algoritmos", etc.
\renewcommand\lstlistlistingname{Códigos}
\renewcommand\lstlistingname{Código}
\renewcommand\lstlistlistingname{Índice de códigos}

\lstlistoflistings
\afterpage{\blankpage}
\addcontentsline{toc}{chapter}{\lstlistlistingname}


% En este documento (de momento) no se ha considerado incluir un índice de algoritmos/pseudocódigos, como el que aparece en \ref{AdditionalLouvain}

%%%%%%%%%%%%%%%%%%%%%%%%%%%%%%%%%%%%%%%%%%%%%%%%%%%%%%%%%%%%%%%%%%%%%%%%%%%%%%%%%%%





%%%%%%%%%%%%%%%%%%%%%%% Cabeceras y pies de página (Opcional) %%%%%%%%%%%%%%%%%%%%%%%

%\setlength{\headheight}{15.2pt}
\pagestyle{fancy}


\renewcommand{\chaptermark}[1]{\markboth{Capítulo \thechapter.\ #1}{}}

\pagestyle{fancy}
\fancyhf{}
\fancyhead[LO]{\leftmark}
\fancyhead[RO]{}
\fancyhead[RE]{\nouppercase\rightmark}
\fancyhead[LE]{}
\fancyfoot[C]{\thepage}

%%%%%%%%%%%%%%%%%%%%%%%%%%%%%%%%%%%%%%%%%%%%%%%%%%%%%%%%%%%%%%%%%%%%%%%%%%%%%%%%%%%%






%%%%%%%%%%%%%%%%%%%%%%%%%%%%%% Capítulos de la memoria %%%%%%%%%%%%%%%%%%%%%%%%%%%%%



% Capítulo 1
\chapter{Introducción}


%%%%%%%%%%%%%%%%%%%%%%%%%%%%%%%%%%%%%%%%%%%%%%%%%%%%%%%%%%%%%%%%%%%%%%%%%%

% Estilo resto de páginas
\pagestyle{fancy}


% Estilo de párrafo de los capítulos
\setlength{\parskip}{0.75em}
\renewcommand{\baselinestretch}{1.25}
% Interlineado simple
\spacing{1}
% Numeración contenido
\pagenumbering{arabic}
\setcounter{page}{1}

%%%%%%%%%%%%%%%%%%%%%%%%%%%%%%%%%%%%%%%%%%%%%%%%%%%%%%%%%%%%%%%%%%%%%%%%%%



Se puede añadir texto antes de empezar la primera sección.


\section{Contexto y alcance}

Contexto. Situar al lector. Objetivo general y alcance del trabajo.


\section{Estructura del documento}

La estructura del TFG no es fija. El tutor indicará una estructura adecuada dependiendo del trabajo concreto.

Se puede incluir dentro de cada apartado secciones adicionales. La copia en papel de la memoria del TFG será encuadernada en pasta dura de color azul (p.e. encuadernación tipo chanel). La portada, que puede ser una pegatina transparente, seguirá el modelo que se adjunta, que incluye el escudo y nombre de la URJC, la titulación cursada por el alumno, el curso académico, el título del TFG, el autor y el o los directores/tutores.


\subsection{Trabajos de grados en informática}

Una posible estructura de la memoria final asociada con cada TFG podría ser la siguiente (leed la normativa de TFG):
\begin{enumerate}
 \item Introducción
 \item Objetivos (incluyendo descripción del problema, estudio de alternativas y metodología empleada)
 \item Descripción informática (puede incluir especificación, diseño, implementación y pruebas).
 \item Experimentos / validación
 \item Conclusiones (incluyendo los logros principales alcanzados y posibles trabajos futuros)
 \item Bibliografía
 \item Apéndices
\end{enumerate}


\subsection{Trabajos del grado en matemáticas}

Una posible estructura de la memoria final asociada con cada TFG podría ser la siguiente:
\begin{enumerate}
 \item Introducción
 \item Objetivos (incluyendo descripción del problema, estudio de alternativas y metodología empleada)
 \item Material y métodos / Metodología / Cuerpo del trabajo (describir las metodologías empleadas en el desarrollo del TFG o el desarrollo del mismo en caso de ser un trabajo de recopilación bibliográfica sobre un tema).
 \item Resultados (opcional, dependiendo del tipo de trabajo desarrollado)
 \item Conclusiones (incluyendo los logros principales alcanzados y posibles trabajos futuros)
 \item Bibliografía
 \item Apéndices
\end{enumerate}


% \afterpage{\blankpage} % puede generar problema en índice de contenidos
% \newpage







% Capítulo 2
\chapter{Objetivos}


Objetivos generales y específicos del trabajo.


\section{Primera sección}

En este capítulo se pueden añadir secciones, pero no son obligatorias en capítulos cortos.

\section{Segunda sección}

El objetivo de este documento es proporcionar una plantilla de \LaTeX para TFG. No debe usarse como sustituto de la normativa de TFG aprobada por la ETSII.


\blankpage






% Capítulo 3
\chapter{Descripción Informática}
\label{chap:contenidos}




\section{Fuentes de datos}

% Citar una referencia
Esto es una referencia bibliográfica \cite{bibex}. Se recomienda leer ``The Not So Short Introduction to \LaTeX'' \cite{Oetiker2007} (existen versiones más modernas).


\section{Programación de queries en PySpark}

Gracias a la ecuación de Euler ($e^{ \pm i\theta } = \cos \theta \pm i\sin \theta$) podemos ver la relación entre varias de las constantes matemáticas más importantes:
\[
    e^{i\pi} + 1 = 0.
\]


% Fórmula numerada
Si una ecuación se va a referenciar es necesario numerarla:
\begin{eqnarray}
\label{eq:schemeP}
 \Phi (k)=\dfrac{2}{|R(k)|(|R(k)|-1)} \underset{i,j \in R(k)}{\sum} a_{ij}.
\end{eqnarray}
Posteriormente se hace referencia a la ecuación a través de su etiqueta (label). Por ejemplo, la anterior ecuación \eqref{eq:schemeP}.



Problema de optimización:
\begin{equation}\label{eq:LP1}
\begin{array}{cl}
  \displaystyle \begin{array}{c}\mathrm{minimizar} \\ \mathbf{t} \in \mathbb{R}^{n}, \  \mathbf{p} \in \mathbb{R}^{m} \end{array} & \hspace{-0.2cm} \begin{array}{c} \mathbf{1}^{\transpuesta}\mathbf{t} \\ \mbox{} \end{array}  \\
  & \vspace{-0.4cm} \\ % línea (fila) en blanco, pero la hacemos estrecha con el comando vspace
  \mbox{sujeto a} & -\mathbf{t} \preceq  \mathbf{V}\mathbf{p} - \mathbf{x}  \preceq  \mathbf{t},\\
 \end{array}
\end{equation}




\section{Programación de queries en Scala/Spark}

% Insertar una tabla
\begin{table}
  \centering
  \caption{Título de la tabla.}
  \label{tab:una_tabla}

\begin{footnotesize}
\renewcommand{\arraystretch}{1.5} % Para cambiar la separación entre filas (1 por defecto)
\begin{tabular}{ccccccccccc}
  \hline
   & Subs. & Students & A & PE & WA & RE & CTE & IF & TLE & All\\
  \hline
Ex. 1 & 104 & 44 & 1.27    &   0       &   0.55    &   0.23    &   0.20    &   0.11    &   0     & 2.36  \\
Ex. 2 & 118 & 37 & 0.92    &   0       &   0.92    &   0.27    &   0.49    &   0.59    &   0     & 3.19  \\
Ex. 3 & 100 & 28 & 1.21    &   0.39    &   1.18    &   0.54    &   0.14    &   0.07    &   0.04  & 3.57  \\
Ex. 4 & 78  & 25 & 1.08    &   0.84    &   0.52    &   0.40    &   0.24    &   0.04    &   0     & 3.12  \\
Ex. 5 & 116 & 31 & 1.48    &   0.10    &   0.77    &   0.32    &   0.42    &   0.19    &   0.45  & 3.74  \\
Ex. 6 & 213 & 32 & 1.06    &   0.34    &   3.81    &   0.56    &   0.69    &   0.06    &   0.13  & 6.66  \\
Ex. 7 & 116 & 34 & 1.35    &   0.38    &   0.38    &   0.68    &   0.62    &   0       &   0     & 3.41  \\
  \hline
Average & 120.7 & 33 & 1.20 &  0.26 &  1.14 &  0.42 &  0.40 &  0.16 &  0.08 & 3.66 \\
  \hline
 \end{tabular}
\end{footnotesize}

\end{table}









\begin{sidewaystable}
  \centering
  \caption{Tabla rotada. Factor groupings for the Mooshak questionnaire.}\label{tab:factor_analysis}

\renewcommand{\arraystretch}{1.1}
\begin{scriptsize}
 \begin{tabular}{clcc}
   \hline
   Factor & \textbf{Interpretation} / Items$^{*}$ (loadings)  & Median & Mode \\
   \hline
   \hline
    1 & \multicolumn{3}{l}{\textbf{Students' perception of Mooshak towards its helpfulness in learning} } \\
   \hline
    (21.17\%) & m10. Mooshak has forced me to implement programs more carefully $(0.849)$ & 4 & 4 \\
    $\alpha$ = 0.922 & m6.  Mooshak has helped me improve as a programmer $(0.819)$ & 3 & 4 \\
     & m5.  Mooshak has made me more aware of the need to write correct code $(0.781)$ & 3 & 3\\
     & m1. Mooshak has forced me to program more responsibly $(0.713)$ & 3 & 3 \\
     & m15. The specifications regarding the exercises used with Mooshak are adequate $(0.687)$ & 3 & 3 \\
     & m18. Mooshak helps to measure my current programming skills $(0.680)$ & 2.5 & 3 \\
   \hline
%   \multicolumn{4}{c}{} \vspace{-0.2cm}\\
%   \hline
    2 & \multicolumn{3}{l}{\textbf{Disposition towards using Mooshak} } \\
   \hline
    (17.93\%) & m24. I would be willing to participate in a programming contest using Mooshak, with similar exercises to the ones & 2 & 1 \\
    $\alpha$ = 0.897 & seen throughout the course $(0.807)$ & & \\
    & m13. Using Moohak in the final exams is a good idea $(0.748)$ & 2 & 1 \\
    & m14. I would like to use Mooshak or a similar tool in the future $(0.734)$ & 3 & 1 \\
    & m17. Knowing Mooshak can motivate me to take part in a programming contest $(0.655)$ & 2 & 1\\
    & m9. It would have been useful to use Mooshak from the first programming course $(0.527)$ & 2.5 & 1\\
     & m16. Using Mooshak in the course has been interesting $(0.522)$ & 3 & 4 \\
   \hline
%   \multicolumn{4}{c}{} \vspace{-0.2cm}\\
%   \hline
    3 & \multicolumn{3}{l}{\textbf{Effect of Mooshak's feedback in the tool's usefulness} } \\
   \hline
    (14.84\%) & m12. Mooshak's feedback is adequate $(0.832)$ & 2 & 1\\
    $\alpha$ = 0.836 & m3. Using Mooshak has increased my workload considerably $(0.693)$ & 4 & 4 \\
     & m7.  If Mooshak does not accept my code I feel motivated to find and fix the errors $(0.691)$ & 2 & 3 \\
     & m8.  In general, using Mooshak has been a good idea $(0.666)$ & 3 & 4 \\
   \hline
%   \multicolumn{4}{c}{} \vspace{-0.2cm}\\
%   \hline
    4 & \multicolumn{3}{l}{\textbf{Mooshak's effect on persistence} } \\
   \hline
    (11.20\%) & m23. When Mooshak does not accept my code I get discouraged and I abandon the exercise $(0.848)$ & 3 & 3 \\
    $\alpha$ = 0.705 & m22. Mooshak has been a waste of time $(0.597)$ & 2 & 2 \\
    & m25. Once a program has passed Mooshak's tests, I rewrite it in order to enhance it $(0.559)$ & 2 & 2 \\
   \hline
%   \multicolumn{4}{c}{} \vspace{-0.2cm}\\
%   \hline
   5 & \multicolumn{3}{l}{\textbf{Students' perception of Mooshak's features} } \\
   \hline
    (10.87\%) & m20. Even if it is not related to the grade, I feel satisfied if I am one of the first students to complete an exercise $(0.729)$ & 2 & 2\\
   $\alpha$ = 0.742  & m19. I value the fact that a tool like Mooshak returns feedback in real time about the correction of my programs $(0.650)$ & 3.5 & 4 \\
   \hline
%   \multicolumn{4}{c}{} \vspace{-0.2cm}\\
%   \hline
\multicolumn{4}{l}{\scriptsize $^{*}$Measured on a 5-point Likert scale (1: strongly disagree; 2: disagree; 3: neutral; 4: agree; 5: strongly agree).}
  \end{tabular}
\end{scriptsize}
\end{sidewaystable}



\begin{table}
  \centering

\begin{small}
\begin{tabular}{|l|l|l|l|}\hline
  \multirow{10}{*}{numeric literals} & \multirow{5}{*}{integers} & in decimal & \verb|8743| \\ \cline{3-4}
  & & \multirow{2}{*}{in octal} & \verb|0o7464| \\ \cline{4-4}
  & & & \verb|0O103| \\ \cline{3-4}
  & & \multirow{2}{*}{in hexadecimal} & \verb|0x5A0FF| \\ \cline{4-4}
  & & & \verb|0xE0F2| \\ \cline{2-4}
  & \multirow{5}{*}{fractionals} & \multirow{5}{*}{in decimal} & \verb|140.58| \\ \cline{4-4}
  & & & \verb|8.04e7| \\ \cline{4-4}
  & & & \verb|0.347E+12| \\ \cline{4-4}
  & & & \verb|5.47E-12| \\ \cline{4-4}
  & & & \verb|47e22| \\ \cline{1-4}
  \multicolumn{3}{|l|}{\multirow{3}{*}{char literals}} & \verb|'H'| \\ \cline{4-4}
  \multicolumn{3}{|l|}{} & \verb|'\n'| \\ \cline{4-4}          %% here
  \multicolumn{3}{|l|}{} & \verb|'\x65'| \\ \cline{1-4}        %% here
  \multicolumn{3}{|l|}{\multirow{2}{*}{string literals}} & \verb|"bom dia"| \\ \cline{4-4}
  \multicolumn{3}{|l|}{} & \verb|"ouro preto\nmg"| \\ \cline{1-4}          %% here
\end{tabular}
\end{small}

  \caption{Tabla con ``multicolumnas'' y ``multifilas''.}\label{tab:tablacompleja}
\end{table}





% Insertar una figura
\begin{figure}
  \centering
  \includegraphics[width=0.75\textwidth,clip=true]{\logoUniversidad}
  \caption{Logo de la Universidad.}
  \label{fig:logo_universidad}
\end{figure}

% Referenciar una etiqueta (label)
Las tablas y figuras deben presentarse en el texto, referenciadas y numeradas. La descripción de una figura debe ir posicionada debajo de la misma. Las descripciones de tablas pueden aparecer encima o debajo de las mismas (pero de forma consistente en todo el documento).

En las tablas se recomienda evitar líneas verticales y usar pocas horizontales. 

La figura~\ref{fig:logo_universidad} se utiliza en la portada. \LaTeX ubica automáticamente las tablas y figuras. Para ello emplea reglas basadas en la experiencia de profesionales de la edición de textos. Podemos forzar su ubicación, pero en general es recomendable usar la ubicación sugerida por el sistema \LaTeX. Usad gráficos vectoriales siempre que podáis.





\begin{figure}
   \centering

  \begin{minipage}{0.45\textwidth}
   \centering


    \footnotesize (a)
  \end{minipage}
  \hfill
  \begin{minipage}{0.45\textwidth}
   \centering

   \footnotesize (b)
  \end{minipage}

    \bigskip

    \footnotesize (c)

  \caption{Ejemplo con varias figuras. Demostración visual del teorema de Pitágoras. En (a) tenemos un triángulo rectángulo con hipotenusa $c$ y catetos $a$ y $b$. En (b) se muestra tres copias escaladas del mismo triángulo. El verde se ha escalado por $a$, el rojo/rosa por $b$, y el azul por $c$. En (c) se juntan los triángulos de (b) para formar un rectángulo cuya base es $c^{2}$, pero también $a^{2} + b^{2}$. Por tanto, $a^{2} + b^{2} = c^{2}$.}\label{fig:teoremapitagoras}
\end{figure}





\section{Despliegue en AWS EMR}

% Nueva página
Normalmente no tendremos que insertar saltos de página, salvo para forzar que los capítulos empiecen en páginas impares, con \begin{verbatim}\blankpage\end{verbatim} En cualquier caso, podemos introducir un salto de página con el comando \begin{verbatim}\newpage\end{verbatim}.

\newpage
% También con \pagebreak



\subsection{Código}


\begin{mypython}[float={!t},caption={Titulo del algoritmo/código.},label={alg:etiqueta}]
def sum_list_limits_1(a, lower, upper):
    if lower > upper:
        return 0
    else:
        return a[upper] + sum_list_limits_1(a, lower, upper - 1)
\end{mypython}
El código~\ref{alg:etiqueta} es un ejemplo en Python.



\begin{algorithm}
\begin{algorithmic}[1]
\STATE $\forall i \in V$, \ let $i$ be an isolated community
\STATE $o=permutation(V)$
\FOR{$k \ \in \ o$}
\STATE search in $A$ all the neighbours of $k$, $j$
\STATE $\forall j$, calculate $\Delta Q_k(j)$ in matrix $\mathcal{M}$
\STATE $j^*=\{ \ j \ | \ \Delta Q_k(j^*)=\max_j\{Q_k(j)\} \ \}$
\IF{$\Delta Q_k(j^*)>0$}
\STATE{Move node $k$ to $j^*$ 's community}
\ELSE
\STATE{$k$ remains in its community}
\ENDIF
\ENDFOR
\end{algorithmic}\caption{\textit{Additional Louvain} \textbf{input}=$\left(A, \ \mathcal{M}\right)$ \textbf{output}=$P$}
\label{alg:AdditionalLouvain}
\end{algorithm}
En el algoritmo~\ref{alg:AdditionalLouvain} aparece un ejemplo en pseudocódigo.











% Nuevo capítulo
\chapter{Experimentos / Validación}
\label{sec:expVal}


En esta sección se describen en detalle las queries realizadas sobre este dataset.


\blankpage

\section{Consultas Realizadas}

\subsection{Descripción General}

\subsubsection{Piloto más consistente de la temporada 2012}
En esta query intentaremos averiguar cuál ha sido el piloto más consistente de la temporada 2012. Ya que este término puede resultar ambigüo, en concreto intentaremos averiguar qué piloto tuvo una menor diferencia entre la media de sus vueltas rápidas del campeonato y la media de todas las vueltas de todos los Grandes Premios de la temporada.

Necesitaremos cruzar varias fuentes de datos para esto: 

\begin{compactitem}
  \item \texttt{races.csv}
  \item \texttt{lap$\_$times.csv}
  \item \texttt{drivers.csv}
  \item \texttt{results.csv}
\end{compactitem}

Para filtrar cualquier query por temporada o por rango de temporadas, tenemos que filtrar la tabla \texttt{races} según la columna \texttt{year}. En nuestro caso, filtramos la tabla para la temporada \texttt{2012}.

Una vez tenemos todas las carreras de la temporada en cuestión, necesitamos hallar el tiempo medio por vuelta de cada piloto en cada Gran Premio. En la tabla \texttt{lap$\_$times.csv} existe una columna llamada \texttt{milliseconds}, que denota los milisegundos de la vuelta dada. Haciendo un right join, podemos obtener todas las vueltas dadas en una temporada. Usando una ventana que particione los datos por piloto, haremos la media de la columna \texttt{milliseconds} para hallar la media de toda la temporada.

Para obtener la media de las vueltas rápidas de cada piloto, debemos usar la tabla \texttt{results}, que tiene una entrada por cada piloto y Gran Premio que contiene además su vuelta más rápida según el formato \texttt{MM:ss.mmm}, siendo \texttt{MM} los minutos, \texttt{ss} los segundos y \texttt{mmm} los milisegundos. Para convertir este formato a milisegundos y viceversa, he creado dos UDFs llamadas \texttt{lapTimeToMs} y \texttt{msTolapTime}.


\begin{lstlisting}
val lapTimeToMs = (time: String) => {
  val regex = """([0-9]|[0-9][0-9]):([0-9][0-9])\.([0-9][0-9][0-9])""".r
  time match {
    case regex(min,sec,ms) => min.toInt * 60 * 1000 + sec.toInt * 1000 + ms.toInt
    case "\\N" => 180000
  }  
}: Long
  \end{lstlisting}

\begin{lstlisting}
val msToLapTime = (time: Long) => {
  val mins = time / 60000
  val secs = (time - mins*60000)/1000
  val ms = time - mins*60000 - secs*1000
    
  val formattedSecs = if((secs / 10).toInt == 0) "0" + secs else secs
  // if ms = 00x -> "0"+"0"+x . if ms = 0xx -> "0"+ms
  val formattedMs = if((ms / 100).toInt == 0) "0" + (if((ms / 10).toInt == 0) "0" + ms else ms) else ms
  mins + ":" + formattedSecs + "." + formattedMs    
}: String
\end{lstlisting}

La función \texttt{lapTimeToMs} debe tener en cuenta también que hay entradas que no tienen un tiempo válido, sino que contienen el string \texttt{"\textbackslash\textbackslash N"}. Esto se debe a que hay ocasiones en los que un piloto no termina la vuelta. Un ejemplo sería si se retirase de la carrera en esa vuelta. Se ha decidido sumar 3 minutos (180000 ms) como penalización.

Una vez tenemos estas dos funciones, ya podemos el tiempo medio de vueltas rápidas. De nuevo particionamos los datos según el piloto y hacemos la media de la columna \texttt{fastestLapTime}. Este valor en milisegundos se convierte al formato de tiempo de vuelta con la UDF correspondiente y tras esto se ordena por el diferencial hallado de forma ascendente.

Además de esto, haciendo un análisis previo se observó que había pilotos que habían completado pocas carreras en comparación al resto, ya que a que sustituían a algún otro piloto debido a algún accidente. Se decidió filtrar estos outliers de forma que, independientemente del periodo de tiempo sobre el que queramos lanzar esta query, el resultado fuese significativo. Para ello se usó la tabla obtenida con todas las carreras en la temporada y, a partir de la columna \texttt{lap} de la tabla \texttt{lap$\_$times} se realizó un conteo de todas las vueltas dadas por cada piloto utilizando, de nuevo, una ventana de datos particionada por piloto.

De este DataFrame queremos obtener tanto la cuenta de vueltas por piloto como el numero total de pilotos que han participado en la temporada, que haremos de la siguiente manera:
\begin{lstlisting}
val (distinctDrivers, allLaps) = lapCount
  .agg(
      countDistinct("driverID"),
      count(col("lap"))
  ).as[(BigInt, BigInt)]
  .collect()(0)

val avgLapsThisPeriod = allLaps.toInt / distinctDrivers.toInt
\end{lstlisting}

Habiendo obtenido la media de vueltas por piloto de la temporada, podemos filtrar aquellos cuyo conteo sea inferior.


Solo quedaría formatear los datos de salida. Para ello primero se hace un join con la tabla de pilotos, lo cual nos permite concatenar nombre y apellido de cada uno de los participantes en una nueva columna. En esta caso, como solo nos interesa el piloto y su diferencial, estas serán las únicas columnas que dejaremos en el DataFrame.

Sobre qué piloto fue el más consistente en la temporada 2012, este análisis nos dice lo siguiente:

INSERTAR IMAGEN DE RESULTADO 

Lo interesante de esta query es que no solo vale para esta temporada. En el dataset que trato se tiene información de tiempos de vuelta desde el año 1996, así que podemos especificar cualquier conjunto de temporadas, ya sean consecutivas, salteadas o solamente una temporada en concreto.

Veamos cuál ha sido el piloto más consistente desde el año 1996:

INSERTAR IMAGEN DEL RESULTADO DESDE 1996 HASTA LA ACTUALIDAD


\subsubsection{Dominio de fabricantes en la década de los 90}

Con esta query se pretende hallar qué fabricante ha sido el más dominante en la década de los 90. En concreto intentaremos hallar el número de mundiales ganados y el número de carreras ganadas.

Se usarán las siguientes fuentes de datos:
\begin{compactitem}
  \item \texttt{races.csv}
  \item \texttt{constructor$\_$standings.csv}
  \item \texttt{constructors.csv}
\end{compactitem}

Al igual que en la query anterior, si queremos fijar nuestra atención en un periodo de tiempo, tenemos que hacerlo filtrando la columna \texttt{year} de la \texttt{races.csv}. En este caso, necesitamos todas las carreras entre el año 1990 y el año 1999.

Una vez obtenidas todas las carreras de la década, tenemos que obtener la última carrera de cada temporada. Esto es debido a que en \newline\texttt{constructor$\_$standings.csv} tenemos la clasificación resultante al final de cada carrera. Para ello, usaremos crearemos una ventana en la que particionaremos los datos por año y que usaremos con la función \texttt{max()} sobre la columna \texttt{round}, que nos indica el índice de la carrera, es decir, la primera carrera de la temporada tendrá \texttt{round === 1} para crear una columna llamada \texttt{max} en la que guardaremos el índice de la última carrera de la temporada. Finalmente, filtraremos los datos para quedarnos con aquellos en los que la columna \texttt{round === max}.

Tras esto último, unimos las tabla \texttt{constructor$\_$standings.csv} con la recién obtenida para quedarnos con los resultados en las últimas carreras y filtramos según la columna \texttt{position === 1} para quedarnos con los ganadores. Teniendo esto, podemos ver también que la columna \texttt{wins} nos proporciona el número de victorias de cada escudería en esa temporada, así que, creando una ventana en la que particionemos por fabricante podemos hallar tanto la suma de victorias como el conteo de apariciones de cada una. 

Pasando ya a la presentación de los datos, se filtrarían los constructores duplicados y se ordenarían los datos según el total de campeonatos ganados primero y, en caso de empate, por número de victorias. Además, se hace un join con la tabla de constructores para obtener su nombre.

INSERTAR CAPTURA DE PANTALLA

Lo interesante de esta query es que se puede usar para cualquier periodo de tiempo. Podemos averiguar por ejemplo qué fabricante ha sido el más dominante en toda la historia de la competición y qué constructor ha dominado ciertos años concretos.

INSERTAR CAPTURA DE PANTALLA DE DOMINIO HISTORICO



\section{Análisis de requisitos no funcionales}






% Nuevo capítulo

\chapter{Conclusiones y trabajos futuros}

En este capítulo se detallan las conclusiones derivadas del TFG y la propuesta de posibles trabajos futuros.

Las citas del texto Autor \cite{giaquinta}, Autor \cite{fortune}, Autor \cite{fortuneB}, Autor \cite{mitchell} y Autor \cite{morrey} deben ir referenciadas en la bibliografia.


\section{Texto de relleno}

\lipsum[1-18]
\blankpage









%%%%%%%%%%%%%%%%%%%%%%%%%%%%%%% Bibliografía %%%%%%%%%%%%%%%%%%%%%%%%%%%%%%%

\phantomsection
\addcontentsline{toc}{chapter}{Bibliografía}

\footnotesize{
%\bibliographystyle{hispa}
\bibliographystyle{IEEEtran}
\bibliography{bibliografia}
}



% No expandir elementos para llenar toda la página
\raggedbottom
\afterpage{\blankpage}

\newpage




%%%%%%%%%%%%%%%%%%%%%%%%%%%%%%% Apéndices %%%%%%%%%%%%%%%%%%%%%%%%%%%%%%%

\appendix

\phantomsection
\addcontentsline{toc}{chapter}{Apéndices}

\mbox{}
\vfill
\begin{center}
\begin{Huge}
\textbf{Apéndices}
\end{Huge}
\end{center}
\vfill
\mbox{}
\thispagestyle{empty}

\newpage
\mbox{}
\thispagestyle{empty}
\newpage


% Primer apéndice
\chapter{Este es el primer apéndice}
\label{sec:apendice}


\section{Ejemplo de sección}

Sección del apéndice




% Fin del documento
\end{document}
